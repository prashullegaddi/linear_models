\documentclass[11pt]{beamer}
\usepackage{latexsym}
\usepackage{wasysym}
\usepackage{algpseudocode}
\usepackage{algorithm2e}
\usepackage{breqn}
%\usefonttheme{serif} % default family is serif
%\usepackage{fontspec}
%\setmainfont{Liberation Serif}


\mode<presentation>
\usefonttheme{professionalfonts}
\usetheme{Frankfurt}
%\usetheme{Warsaw}
%\usetheme{Darmstadt}
%\usecolortheme{whale}


\title{linear models notes}
%\author[Prashant]{prashant ullegaddi %\\ 
%%\texttt{prashant.ullegaddi@research.iiit.ac.in}  
%}
%   \href{mailto:prashant.ullegaddi@research.iiit.ac.in}{\texttt{prashant.ullegaddi@research.iiit.ac.in}}}
\begin{document}
\begin{frame}
%	\date{\today}
	\titlepage
\end{frame}


\section{Linear Regression}
\subsection{problem}
\begin{frame}
	\frametitle{\textbf{linear regression}}
	\begin{block}{What's the problem?}
	\begin{itemize}
		\item Given a set of $m$ training examples $\{X^{(i)}, y^{(i)}\}_{i=1}^m$ 
		where $X^{(i)} \in \mathbb{R}^{n+1}$
		with $X^{(i)}_0 = 1$ and $y^{(i)} \in \mathbb{R}$.
		\item Learn a hypothesis function $h(\cdot)$ such that for any new test example $x^{(t)}$, $y$ can be predicted as 
		$y^{(t)} = h(x^{(t)})$.
		\item Fit a hyperplane $h_\theta(x) = \theta_0 + \theta_1 x_1 + \cdots + \theta_n x_n = \sum_{i=0}^n \theta_i x_i = \theta^T x$
	\end{itemize}
	\end{block}
\end{frame}

\subsection{ordinary-gradient method}
\begin{frame}
	\frametitle{ordinary least squares method}
	\begin{block}{Optimization}
	Find parameters $\theta$ that:
	\begin{equation*}	
		\textrm{minimize}_\theta J(\theta) = \frac{1}{2} \sum_{i=1}^{m} (h_\theta (X^{(i)}) - y^{(i)})^2
	\end{equation*} 
   	\end{block}
\end{frame}

\begin{frame}
	\frametitle{gradient descent method}
	\begin{block}{Update rule}
     	\begin{equation*}
		\theta := \theta - \alpha \nabla_\theta J(\theta)
	\end{equation*}
	\indent where $\alpha$ is the learning rate, 
	$\nabla_\theta J(\theta)$ is the gradient of cost function $J(\theta)$.
   	\end{block}
\end{frame}

\begin{frame}
	\frametitle{derivation of gradient} 
	\begin{align*} 
	 J(\theta) &= \frac{1}{2} \sum_{i=1}^{m} (h_\theta (X^{(i)}) - y^{(i)})^2 \\
	\frac{\partial}{\partial \theta_j} J(\theta) &= \sum_{i=1}^{m} (h_\theta (X^{(i)}) - y^{(i)}) \frac{\partial}{\partial \theta_j} (\theta^T X^{(i)} - y^{(i)})
	\end{align*}
	\begin{block}{Gradient}
	\begin{align*}
	\frac{\partial}{\partial \theta_j} J(\theta) &= \sum_{i=1}^{m} (h_\theta (X^{(i)}) - y^{(i)}) X^{(i)}_j \\ 
	\nabla_\theta J(\theta)  &= 	\begin{bmatrix}
	\frac{\partial}{\partial \theta_0} J(\theta) & \frac{\partial}{\partial \theta_1} J(\theta)  & \cdots & \frac{\partial}{\partial \theta_n} J(\theta) 
								\end{bmatrix}^{T}
	\end{align*}
	\end{block}
\end{frame}

\subsection{stochastic gradient}
\begin{frame}
	\frametitle{stochastic gradient method}
	
\begin{algorithm}[H]
 	\SetAlgoLined
 	\KwData{$\alpha$, $m$ training examples $\{X^{(i)}, y^{(i)}\}_{i=1}^m$} 
	\KwResult{$\theta$ parameters}
	$\theta := \mathbf{0}$ \;
	\For{$i = 1 \cdots m$} {
		\For{$j = 0 \cdots n$} {
			$\theta := \theta - \alpha (h_\theta (X^{(i)}) - y^{(i)}) X^{(i)}_j$  \;
		}
	}
 \caption{Stochastic gradient algorithm}
\end{algorithm}
	
\end{frame}

\subsection{closed form solution}
\begin{frame}
	\frametitle{closed form: add derivation involving trace}

\end{frame}

\begin{frame}
	\frametitle{closed form solution}
	\begin{block}{Solution}
	\begin{equation*}	
		\theta = (X^T X)^{-1} X^T y 
	\end{equation*} 

	where $X$ is an $m \times (n+1)$ data matrix with $X[:,1] = 1$ and $y$ is a $m \times 1$ vector of labels
   	\end{block}
\end{frame}

\section{Logistic Regression}
\subsection{what it is}
\begin{frame}
	\frametitle{\textbf{logistic regression}}
	
	\begin{block}{What's the problem?}
	\begin{itemize}
		\item Given a set of $m$ training examples $\{X^{(i)}, y^{(i)}\}_{i=1}^m$ 
		where $X^{(i)} \in \mathbb{R}^{n+1}$
		with $X^{(i)}_0 = 1$ and $y^{(i)} \in \{0, 1\}$ is a label.
		\item Learn a hypothesis function $h(\cdot)$ such that for any new test example $x^{(t)}$, label for it can be predicted as 
		$y^{(t)} = h(x^{(t)})$.
		\item In logistic regression, we assume:
		\begin{equation*}
		h(x) = \textrm{sigmoid}(\theta^T x) = \frac{1}{1 + e^{-\theta^T x}}
		\end{equation*}
	\end{itemize}
	\end{block}
\end{frame}

\subsection{sigmoid}
\begin{frame}
	\frametitle{property of sigmoid function}
	\begin{align*}
	\textrm{Let } g(x) &= \frac{1}{1+e^{-x}} \\
	\textrm{Then, } g'(x) &= -\frac{1}{(1+e^{-x})^2}\frac{d}{dx}(1+e^{-x})\\
		 &= \frac{e^{-x}}{(1+e^{-x})^2} \\
		&= \left(\frac{1}{1+e^{-x}}\right) \left(\frac{e^{-x}}{1+e^{-x}}\right) \\
		&= \left(\frac{1}{1+e^{-x}}\right) \left(1 - \frac{1}{1+e^{-x}}\right) 
	\end{align*}
\vspace{-10pt}
	\begin{block}{Thus,}
	\begin{equation}
	\label{eq:sigmoid_derivative}
	g'(x) = g(x) (1-g(x))
	\end{equation}
	\end{block}
\end{frame}

\subsection{how}
\begin{frame}
	\frametitle{optimization---maximize likelihood of the parameters $\theta$}
		Assume  
		\begin{align*}
		p(y=1|x; \theta) &= h_\theta(x)\\
		p(y=0|x; \theta) &= 1-h_\theta(x) \\
		\textrm{or more generally}\\
		p(y|x; \theta) &= (h_\theta(x))^y (1-h_\theta(x))^{(1-y)}
		\end{align*}

		\begin{block}{Likelihood of parameters}
		\begin{align*}
			L(\theta) &= p(\mathbf{y} | X; \theta) = \prod_{i=1}^{m} p(y^{(i)} | X^{(i)}; \theta) \\
					 &= \prod_{i=1}^{m} (h_\theta(X^{(i)}))^{y^{(i)}} (1-h_\theta(X^{(i)}))^{(1-y^{(i)})}
		\end{align*}
		\end{block}
\end{frame}

\begin{frame}
	\frametitle{or equivalently maximize loglikelihood of the parameters $\theta$}
	It's easier to maximize loglikelihood $\log(L(\theta))$ which is the same as maximizing likelihood $L(\theta)$
		\begin{block}{Optimization}
		\begin{align*}
			l(\theta) &= \log{L(\theta)} \\
					 &= \sum_{i=1}^{m} [y^{(i)} \log{h_\theta(X^{(i)})} + (1-y^{(i)}) \log{(1-h_\theta(X^{(i)}))}] \\
			& \max_\theta l(\theta)
		\end{align*}
		\end{block}

		\begin{block}{Gradient ascent}
		Update rule:
		\begin{align*}
			\theta := \theta + \alpha \nabla_\theta l(\theta)
		\end{align*}
		\end{block}
\end{frame}

\begin{frame}
	\frametitle{derivation of gradient of $l(\theta)$}
	\begin{align*}
	\frac{\partial}{\partial \theta_j} l(\theta) &= \sum_{i=1}^{m} [y^{(i)} \log{h_\theta(X^{(i)})} + (1-y^{(i)}) \log{(1-h_\theta(X^{(i)}))}] \\
			&= \sum_{i=1}^{m} [y^{(i)} \frac{1}{h_\theta(X^{(i)})} \frac{\partial}{\partial \theta_j} h_\theta(X^{(i)}) \\
			&+ (1-y^{(i)}) \frac{1}{(1-h_\theta(X^{(i)}))} \frac{\partial}{\partial \theta_j} (1-h_\theta(X^{(i)})) ] \\
			&= \sum_{i=1}^{m} [y^{(i)} \frac{1}{h_\theta(X^{(i)})}  h'_\theta(X^{(i)}) \frac{\partial}{\partial \theta_j}\theta^T X^{(i)} \\
			&+ (1-y^{(i)}) \frac{1}{(1-h_\theta(X^{(i)}))}  (-h'_\theta(X^{(i)})) \frac{\partial}{\partial \theta_j} \theta^T X^{(i)} ]
	\end{align*}
\end{frame}

\begin{frame}
	\frametitle{derivation of gradient of $l(\theta)$}
	\begin{center}
	\begin{align*}
	\noindent \textrm{Using Eq. \ref{eq:sigmoid_derivative}, } \\
	\scriptstyle \frac{\partial}{\partial \theta_j} l(\theta) 
		 &= \scriptstyle \sum_{i=1}^{m} [y^{(i)} \frac{1}{h_\theta(X^{(i)})}  h_\theta(X^{(i)}) (1-h_\theta( X^{(i)}) \frac{\partial}{\partial \theta_j}\theta^T X^{(i)} \\
	\scriptstyle	&+\scriptstyle (1-y^{(i)}) \frac{1}{(1-h_\theta(X^{(i)}))}  (-h_\theta(X^{(i)}) (1-h_\theta(X^{(i)}))) \frac{\partial}{\partial \theta_j} \theta^T X^{(i)} ] \\
	\scriptstyle	&= \scriptstyle \sum_{i=1}^{m} [y^{(i)}  (1-h_\theta( X^{(i)}) X^{(i)}_j + (1-y^{(i)})  (-h_\theta(X^{(i)})  X^{(i)}_j ]	
		%&= \sum_{i=1}^{m} (y^{(i)} -h_\theta(X^{(i)}))  X^{(i)}_j 
	\end{align*} \vspace{-20pt}
	\begin{block}{Derviative}
	\begin{equation*}
		\frac{\partial}{\partial \theta_j} l(\theta) = \sum_{i=1}^{m} (y^{(i)} -h_\theta(X^{(i)}))  X^{(i)}_j 
	\end{equation*}
	\end{block}
	\end{center}
\end{frame}

\end{document}
